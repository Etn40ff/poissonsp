\documentclass{article}

\begin{document}
Given an aperiodic BD triple for $SP_{2n}$, our goals:
\begin{enumerate}
    \item Define a (non standard) cluster structure on $SP_{2n}$.
    Strategy:
\begin{itemize}
    \item 'Unfold' the BD triple to a BD triple for $SL_{2n}$.
    \item Let $\Sigma$ be an initial seed in the GSV construction.
    \item Temporarily freeze the cluster variables in $\Sigma$ 
    that correspond to frozen ones in BFZ to get a new seed $\Sigma\prime$.
    \item Show that $h^*$ gives a quasi isomorphism (Fraser) from $\Sigma\prime$ 
    to $\Sigma_{BFZ}$. We want to say that $h^*(\Sigma)$ is a product of minors times frozen, 
    the $B$ matrix is the same, and $h^*$ is invertible.
    % Fraser Proposition 3.2 and Lemma 3.7
    \item In the standard structure we can find a foldable word.
    We need to define "foldable" under the "star" involution,
    and then show the following:
    \begin{itemize}
        \item folding of the word gives a reduced word for the longest element of $SP_{2n}$.
        \item we get a generic element $g$ of $SP_{2n}$.
        \item $\Delta_{\mu}^{\lambda}(g) = \Delta_{\mu^*}^{\lambda^*}(g)$.
        \item in the standard seed for the foldable word we have both $\Delta_{\mu}^{\lambda}$ and $\Delta_{\mu^*}^{\lambda^*}$ 
    \end{itemize}
    \item observe that $h^*$ and mutations commute so get a foldable seed for the non-standard structure.
    (Discuss what happens to arrows in between frozen variables.
    In particular we need to discuss what happens when we unfreze the variables we temporarely froze.)
    \item Show that $h^*$ and folding commute, so we get the seed we wanted.
\end{itemize}
    \item Show it is regular. \\
    If we new that folding "works" on the level of cluster algebras, 
    this would be for free. Alternatively, we use the starfish lemma (which should be easy).

    \item Prove that the upper cluster algebra is the whole ring.
    \begin{itemize}
        \item To show that generators of the ring of coordinates of $SP$ are in the upper cluster algebra, i.e., they are Laurent polynomials
        in all seeds.
        \item by GSV (4.17 in "unified approach", see also "plethora") it suffices to show that generators of $SP$ are Laurent polynomials in the initial cluster and its neighbours.
        \item Observe that these clusters are the folding of $SL$ clusters.
        \item Each generator of $SP$ is the folding of a generator of $SL$.
        \item Each generator of $SL$ is a Laurent polynomial in every cluster of $SL$, in particularin the one we need.
        \item Folding sends Laurent polynomials to Laurent polynomials.
    \end{itemize}
\end{enumerate}



    
\end{document}
